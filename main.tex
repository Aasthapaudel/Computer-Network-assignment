
\documentclass[12pt]{article}

\usepackage[margin=1in]{geometry}
\usepackage{amsmath,amssymb}
\usepackage{graphicx}
\usepackage{hyperref}
\usepackage{listings}
\usepackage[utf8]{inputenc}
\usepackage[english]{babel}
\usepackage{caption}
\usepackage{float}
\usepackage{booktabs}
\usepackage{titlesec}
\usepackage{fancyhdr}

\titleformat{\section}
  {\normalfont\fontsize{14}{15}\bfseries}{\thesection}{1em}{}

\pagestyle{fancy}
\fancyhf{}
\rhead{Your Name}
\lhead{Computer Networks Assignment 1}
\cfoot{\thepage}

\lstset{
  basicstyle=\ttfamily\small,
  breaklines=true,
  columns=fullflexible,
  frame=single,
  captionpos=b,
  language=bash
}

\begin{document}

\begin{titlepage}
\centering
\vspace*{1in}
{\Huge\bfseries Computer Networks Assignment 1\par}
\vspace{1in}
{\Large\bfseries Aastha Paudel(BCT002) \\
\par}
\vfill
\today
\end{titlepage}

\clearpage
\tableofcontents

\section{Introduction}
This report presents a detailed analysis of the concepts covered in Assignment 1, with a focus on IP addressing, subnetting, network configuration, and the practical application of tools like \texttt{ipconfig} and Packet Tracer.

\section{Questions and Answers}

\subsection*{Question 1}
The command \texttt{ipconfig} is used to display the network configuration of the machine. In this case, the output reveals the following relevant information:

\begin{itemize}
\item \textbf{IP Address:} 192.168.1.66
\item \textbf{Subnet Mask:} 255.255.255.0
\end{itemize}

To determine the number of usable IP addresses in this network, we need to understand subnetting. The subnet mask 255.255.255.0, in binary form, is 11111111.11111111.11111111.00000000. The '1' bits represent the network portion, while the '0' bits represent the host portion of the address.

With 8 bits allocated for hosts (the eight '0' bits), we have a maximum of $2^8 = 256$ possible addresses. However, two of these addresses are reserved:

\begin{itemize}
\item \textbf{Network Address:} The address with all host bits set to 0 (e.g., 192.168.1.0) is the identifier for the entire network.
\item \textbf{Broadcast Address:} The address with all host bits set to 1 (e.g., 192.168.1.255) is used to send messages to all devices on the network.
\end{itemize}

Therefore, the number of usable IP addresses in this network is $2^8 - 2 = \boxed{254}$.

\textbf{Packet Tracer Simulation:}

A network simulation was created in Packet Tracer using the obtained IP address (192.168.1.66) and subnet mask (255.255.255.0). Multiple devices were connected, each assigned a unique IP address within the 192.168.1.0/24 range, demonstrating the subnet's capacity.

\begin{figure}[H]
    \centering
    \includegraphics[width=0.8\textwidth]{your_packet_tracer_screenshot.png}
    \caption{Packet Tracer simulation depicting the network configuration.}
\end{figure}

\subsection*{Question 2}
The Packet Tracer simulation (refer to the figure above) visually confirms the network configuration derived from the \texttt{ipconfig} output. It demonstrates how the subnet mask restricts the range of usable IP addresses within the 192.168.1.0 network.

\section{\texttt{ipconfig} Output}

\begin{lstlisting}[language=bash]
Windows IP Configuration

Ethernet adapter wifi:
   Media State . . . . . . . . . . . : Media disconnected
   Connection-specific DNS Suffix  . :

Ethernet adapter VirtualBox Host-Only Network:
   Connection-specific DNS Suffix  . :
   Link-local IPv6 Address . . . . . : fe80::414c:2345:63c2:f25%18
   IPv4 Address. . . . . . . . . . . : 192.168.56.1
   Subnet Mask . . . . . . . . . . . : 255.255.255.0
   Default Gateway . . . . . . . . . :

Wireless LAN adapter Local Area Connection* 1:
   Media State . . . . . . . . . . . : Media disconnected
   Connection-specific DNS Suffix  . :

Wireless LAN adapter Wi-Fi:
   Connection-specific DNS Suffix  . : worldlink.com.np
   IPv6 Address. . . . . . . . . . . : 2400:1a00:bde0:8702::2
   IPv6 Address. . . . . . . . . . . : 2400:1a00:bde0:8702:13f9:586f:fb22:386e
   Temporary IPv6 Address. . . . . . : 2400:1a00:bde0:8702:5f:4a16:755e:4aef
   Link-local IPv6 Address . . . . . : fe80::4f64:55b4:198c:6b5e%13
   IPv4 Address. . . . . . . . . . . : 192.168.1.66
   Subnet Mask . . . . . . . . . . . : 255.255.255.0
   Default Gateway . . . . . . . . . : fe80::1%13
                                       192.168.1.254
\end{lstlisting}

\section{Conclusion}
This assignment provided valuable insights into the core concepts of IP addressing, subnetting, and network configuration. The use of tools like \texttt{ipconfig} and Packet Tracer facilitated a hands-on understanding of these principles, solidifying the theoretical knowledge gained in the course.
\end{document}

